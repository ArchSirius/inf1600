\documentclass[10pt,letterpaper]{article}

%Langue et caractères spéciaux
\usepackage[french]{babel} 
\usepackage[T1]{fontenc}
\usepackage{lmodern}
\usepackage[utf8]{inputenc}
%Marges
\usepackage[top=2cm, bottom=2cm, left=2cm, right=2cm, columnsep=20pt]{geometry}
%Math
\usepackage{amsmath}
\usepackage{textcomp}
\usepackage{multirow}
\usepackage{subfig}
%Graphiques
\usepackage{graphicx}
\usepackage{xcolor}
\definecolor{light-gray}{gray}{0.8}
\usepackage{adjustbox}
\usepackage{verbatim}
\usepackage{float}
%Titres de section/sous-section
\usepackage[compact]{titlesec}
\titleformat{\section}{\large\bfseries\itshape}{\thesection}{1em}{}
\titleformat{\subsection}{\normalsize\bfseries}{\thesubsection}{1em}{}
\titleformat{\subsubsection}{\normalsize}{\thesubsubsection}{1em}{}
\renewcommand{\thesection}{Exercice \arabic{section} :}
\renewcommand{\thesubsection}{\alph{subsection})}
\renewcommand{\thesubsubsection}{\arabic{subsubsection}.}
%Commandes personnalisées
\newcommand{\CRR}[1]{\multirow{2}{*}{#1}}
\newcommand{\GBG}[1]{\fcolorbox{light-gray}{light-gray}{#1}}
\newenvironment{code}%
   {\par\noindent\adjustbox{margin=1ex,bgcolor=light-gray,margin=0ex \medskipamount}\bgroup\minipage\linewidth\verbatim}%
   {\endverbatim\endminipage\egroup}

%Titre
\title{\vspace{-10mm}\Large
INF1600 - TP 2
\vspace{-4mm}}

%Auteur
\author{\large
Samuel Rondeau\\
Pacôme Bondet}
\date{\today}


\begin{document}

\maketitle

\section{architecture avec microcodes}
\subsection{recherche d'instruction}
Donnez la séquence de microcodes qui correspond à la recherche d’une instruction. Donnez la réponse sous la forme d'un tableau. Le tableau suivant montre un exemple de ce qui est attendu (la première ligne étant déjà écrite pour vous) :

\begin{table}[ht]
\centering
\begin{tabular}{| l || c | c | c | c | c | c | c | c | c | c | c | c | c | c | c | c || l |}
\hline
\textbf{RTN concret} & 15 & 14 & 13 & 12 & 11 & 10 & 9 & 8 & 7 & 6 & 5 & 4 & 3 & 2 & 1 & 0 & \textbf{hexa} \\
\hline
\GBG{MA $\gets$ PC;} & 0 & 0 & 1 & 1 & 0 & 0 & 0 & 0 & 0 & 1 & 1 & 0 & 0 & 0 & 0 & 0 & 0x3060 \\
\hline
\GBG{MD $\gets$ M[MA]:} & \CRR{0} & \CRR{1} & \CRR{1} & \CRR{0} & \CRR{1} & \CRR{1} & \CRR{0} & \CRR{0} & \CRR{1} & \CRR{1} & \CRR{0} & \CRR{0} & \CRR{0} & \CRR{0} & \CRR{0} & \CRR{0} & \CRR{0x6CC0} \\
\GBG{PC $\gets$ PC + 4;} & & & & & & & & & & & & & & & & & \\
\hline
\GBG{IR $\gets$ MD;} & 1 & 0 & 0 & 0 & 0 & 0 & 1 & 0 & 0 & 1 & 1 & 0 & 0 & 0 & 0 & 0 & 0x8260 \\
\hline
\end{tabular}
\end{table}

\bigskip
\subsection{instruction générique}
Sous la même forme qu’en a), écrivez la séquence de microcodes permettant d’exécuter l’instruction d’opérations arithmétiques/logiques typiques décrite par le RTN abstrait :
\begin{code}
(IR<31..27> = opcode) ->
	R[IR<26..22>] <- R[IR<21..17>] oper M[R[IR<16..12>] + IR<11..0>];
\end{code}
\hfill
où \GBG{opcode} correspond au code d’opération requis pour exécuter l’opération arithmétique/logique \GBG{oper}. \\
N’oubliez pas d’inclure le RTN concret à chaque ligne du tableau pour justifier vos choix de signaux de contrôle.

\begin{table}[ht]
\centering
\begin{tabular}{| l || c | c | c | c | c | c | c | c | c | c | c | c | c | c | c | c || l |}
\hline
\textbf{RTN concret} & 15 & 14 & 13 & 12 & 11 & 10 & 9 & 8 & 7 & 6 & 5 & 4 & 3 & 2 & 1 & 0 & \textbf{hexa} \\
\hline
\GBG{A $\gets$ R[rc];} & 0 & 0 & 0 & 0 & 0 & 0 & 0 & 0 & 0 & 1 & 1 & 0 & 1 & 1 & 1 & 0 & 0x006E \\
\hline
\GBG{MA $\gets$ A + const;} & 0 & 0 & 0 & 1 & 0 & 0 & 0 & 0 & 0 & 0 & 1 & 0 & 0 & 0 & 0 & 1 & 0x1021 \\
\hline
\GBG{MD $\gets$ M[MA]:} & \CRR{0} & \CRR{0} & \CRR{0} & \CRR{0} & \CRR{1} & \CRR{1} & \CRR{0} & \CRR{0} & \CRR{1} & \CRR{1} & \CRR{1} & \CRR{0} & \CRR{1} & \CRR{0} & \CRR{1} & \CRR{0} & \CRR{0x0CEA} \\
\GBG{A $\gets$ R[rb];} & & & & & & & & & & & & & & & & & \\
\hline
\GBG{R[ra] $\gets$ A oper MD;} & 1 & 0 & 0 & 0 & 0 & 0 & 1 & 0 & 0 & 0 & 0 & 1 & 0 & 0 & 0 & 0 & 0x8210 \\
\hline
\end{tabular}
\end{table}

\bigskip
\subsection{simulation}
Prenez une capture d’écran et intégrez-la dans votre rapport. Cette capture doit 
bien montrer le résultat placé dans \GBG{R[1]} (\GBG{ECX} dans Electric) après l’exécution de 
l’instruction à l’adresse 8 (dans \GBG{tp2mem.txt}, donc après la 3ième instruction). 
Justifiez également le résultat obtenu dans votre rapport. 
Pour rappel, l'adresse présentement exécutée est contenue dans le registre PC. 
Votre capture d'écran doit donc montrer les cycles pendant lesquels PC vaut 8 (PC
est affiché dans la cinquième ligne de la simulation).

\begin{figure}[H]
\centering
\includegraphics[width=\textwidth]{Capture1.png}
\includegraphics[width=\textwidth]{Capture2.png}
\caption{\label{fig:1C} Captures d'écran pour 1.c). La figure du haut représente la section de gauche (à partir de 0 ns) et la figure du bas représente la section de droite (jusqu'à 350 ns). La ligne pointillée montre le moment où \GBG{PC} cesse de valoir 8.}
\end{figure}

\bigskip
\subsection{opération OR}
Soit l’UAL définie dans la cellule ALU de la librairie du TP2 sur Electric. Quelle doit 
être la valeur de \GBG{op[6:0]} pour que l’opération finale soit un OR (c'est-à-dire un OU 
logique bit à bit) ? \\
Vous devez naviguer dans Electric (CTRL+D et CTRL+U) pour répondre à cette 
question. \\
Écrivez cette valeur à l’endroit approprié du fichier \GBG{tp2opalu.txt} et relancez la 
simulation (redémarrez Electric pour être certain) afin de tester l’instruction OR de 
l’adresse 0xc dans \GBG{tp2mem.txt}. \\
Donnez aussi cette valeur dans votre rapport. \\
\par Étudions le code \GBG{op} de \GBG{ADD} dans \GBG{tp2opalu.txt} ainsi que l'ALU afin d'en comprendre le comportement et de trouver le bon code pour \GBG{OR}. Le bit \GBG{op[6]} doit être 0 puisque l'opération ne nécessite pas \GBG{and32}. Le bit \GBG{op[5]} est aussi à 0 car on n'utilise pas \GBG{add32}. Le bit \GBG{op[4]} est aussi à 0 car on n'utilise pas \GBG{shift32} mais plutôt la branche 0 du multiplexeur. Ensuite, étudions la table de vérité de \GBG{AND}, pour vérifier comment obtenir \GBG{op[3:0]} = 1010, soit A. On remarque que \GBG{op[3:0]} est écrit selon la table de vérité suivant le Code Gray, le bit de poids fort étant celui du bas. Nous devrons en tenir compte pour déterminer \GBG{op[3:0]} de \GBG{OR}, qui doit alors être \GBG{op} = 0000 1110, soit \textbf{0E}. \\
La simulation dans Electric est montrée à la figure \ref{fig:1D}.

\begin{table}[ht]
\centering
\subfloat[\label{tab:tt_add_a} Usuelle]{
\begin{tabular}{| c | c | c |}
\hline
A & B & A $\oplus$ B \\
\hline
0 & 0 & 0 \\
0 & 1 & 1 \\
1 & 0 & 1 \\
1 & 1 & 0 \\
\hline
\end{tabular}}
\qquad\qquad
\subfloat[\label{tab:tt_add_b} Code Gray]{
\begin{tabular}{| c | c | c |}
\hline
A & B & A $\oplus$ B \\
\hline
0 & 0 & 0 \\
0 & 1 & 1 \\
1 & 1 & 0 \\
1 & 0 & 1 \\
\hline
\end{tabular}}
\caption{\label{tab:tt_add} Table de vérité de l'additionneur binaire (\GBG{xor})}
\end{table}

\begin{table}[ht]
\centering
\begin{tabular}{| c | c | c |}
\hline
A & B & A | B \\
\hline
0 & 0 & 0 \\
0 & 1 & 1 \\
1 & 1 & 1 \\
1 & 0 & 1 \\
\hline
\end{tabular}
\caption{\label{tab:tt_add} Table de vérité de l'opération \GBG{or} binaire}
\end{table}

% Figure plus bas pour bon format

\bigskip
\subsection{questions}
\subsubsection{Pour l’instruction 0x1234567 du processeur étudié dans ce TP, à quoi servent
les données des deux derniers octets (0x4567) ? Donnez une autre 
instruction (sur 32 bits) qui aurait exactement le même effet d’exécution.}
Les deux derniers octets correspondent à \GBG{IR<15..0>}, qui selon l'architecture, correspondent à la constante (inutilisée) et une partie de \GBG{rc}. \\
0x1234567 = 0x01 0x23 0x45 0x67. On remarque alors qu'il s'agit de l'instruction \GBG{NOP} (opcpde 0) de \GBG{tp2mem.txt}. En effet, l'opcode vaut \GBG{IR<31..27> = 0}. Ainsi, la table de vérité \GBG{op<3..0>} vaut toujours 0, et les autres bits de l'opcode aussi. Ainsi, l'\GBG{UAL} n'effectue aucune opération. On pourrait donc écrire l'instruction 0x0 qui ferait la même chose, rien.

\medskip
\subsubsection{Nommez un avantage d’avoir une architecture à deux bus. Vous êtes -vous 
servi de cet avantage dans votre microprogramme développé en b) ?}
Il est possible d'effectuer plus d'instructions en parallèlle, c'est-à-dire envoyer une information sur le bus A et une information sur le bus B pour faire deux opérations à la fois. C'est ce que nous avons fait à l'étape \GBG{MD $\gets$ M[MA]: A $\gets$ R[rb];} au b).

\medskip
\subsubsection{Diriez-vous que les instructions de cette architecture peuvent être aussi 
flexibles, en terme d’opérations arithmétiques/logiques, que celles du 
processeur étudié à l’exercice 5 du TP1 ? Pourquoi ?}
Oui, voire même plus. Elle contient deux bus, ce qui permet plus d'opérations en parallèle. Elle contient également plus de registres temporaires au lieu de T, pour stocker les données à envoyer à l'UAL. De plus, l'accès à la mémoire est détachée de l'architecture principale, et ne requiert donc ni le bus A ni le bus B.
\vspace{5cm}

\begin{figure}[H]
\centering
\includegraphics[width=\textwidth]{Capture3.png}
\caption{\label{fig:1D} Capture d'écran pour 1.d). La première ligne pointillée montre le moment où les 2 opérations \GBG{AND} sont effectuées. La seconde ligne pointillée montre le moment où l'opération \GBG{OR} est effectuée. Chaque opération correspond au bon \GBG{PC} décalé, puisque \GBG{PC <- PC + 4}.}
\end{figure}

\newpage
\section{assembleur avec processeur à pile}
Écrivez l'expression suivante en assembleur :
\begin{equation*}
a = \left( \frac{b \times c}{f + c} \right) \left( \frac{g - d}{e} \right) + e \times \left( g - d \right)
\end{equation*}

Dans le code en assembleur, nul besoin de déclarer des étiquettes, car \GBG{a}, \GBG{b}, \GBG{c}, \GBG{d}, \GBG{e}, \GBG{f} et \GBG{g} sont déclarées commes variables globales dans la fonction principale. Ainsi, on peut tout simplement écrire le code assembleur suivant qui donne le même résultat que l'instruction en C.
\begin{verbatim}
	.global func_s
	
	func_s:
	    flds b
	    flds c
	    fmulp
	    fstps a    #a = b * c
	    flds f
	    flds c
	    faddp
	    flds a
	    fdivp
	    fstps a    #a = (b * c) / (f + c)
	    flds g
	    flds d
	    fsubrp
	    flds e
	    fdivrp
	    flds a
	    fmulp
	    fstps a    #a = ((b * c) / (f + c)) * ((g - d) / e)
	    flds g
	    flds d
	    fsubrp
	    flds e
	    fmulp
	    flds a
	    faddp
	    fstps a    #a = (((b * c) / (f + c)) * ((g - d) / e)) + (e * (g - d))
	
	
	    ret
\end{verbatim}

\newpage
\section{conditions et branchements}
Écrivez la séquence suivante décrite en langage C :
\begin{verbatim}
	a = b;
	if (c + 1600 > e + 2013) {
	    a = c;
	    if ((b <= c) || (d == e)) {
		        a = e;
	    }
	} else {
	    a = a + b;
	}

\end{verbatim}
où \GBG{a}, \GBG{b}, \GBG{c}, \GBG{d} et \GBG{e} sont des entiers signés (type \textit{int} en langage C) sur 32 bits. Vous pouvez utiliser directement ces symboles pour représenter leurs adresses en assembleur, comme à l'exercice 2.
\begin{verbatim}
	.global func_s
	
	func_s:
	    mov b, %ebx
	    mov %ebx, a        #a = b
	    mov c, %ecx
	    add $1600, %ecx    #(c + 1600)
	    mov e, %edx
	    add $2013, %edx    #(e + 2013)
	    cmp %ecx, %edx
	    jae else           #si (c + 1600 <= e + 2013) sauter au else
	    mov c, %ecx
	    mov %ecx, a        # a = c
	    cmp %ebx, %ecx
	    jae if2            #si (b <= c) entrer dans le if
	    mov d, %edx
	    mov e, %ecx
	    cmp %edx, %ecx
	    je if2             #sinon, si (d == e) entrer dans le if
	    jmp fin            #sinon, quitter
	
	if2:
	    mov e, %ecx
	    mov %ecx,a         #a = e
	    jmp fin

	else:
	    mov a, %eax
	    add b, %eax        #a = a + b
	    mov %eax, a

	fin:
	    ret
\end{verbatim}

\end{document}